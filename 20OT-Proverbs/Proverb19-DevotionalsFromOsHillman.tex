\subsection{Devotions from Os Hillman}

%%%%%%%%%%%%%%%%%%%%%%%%%%%%%%%%%%%%%%%%%%%%%%%%%%%%%%%%%%%%
\subsubsection{Proverb 19:17 - 2021/12/21}

\index[speaker]{Os Hillman!Proverb 19:17 (2021/12/21)}
\index[series]{Proverbs (Os Hillman)!Pro 19:17 (2021/12/21)}
\index[date]{2021/12/21!Proverb 19:17 2021/12/21) (Os Hillman)}

\index[DEVOTIONAL]{TGIF1!Os Hillman (2021/12/21) - Proverb 19:17!2021/12/21}

If you were God and you wanted to send one of your servants to help the less fortunate in the world, how would you train your servant for this task? Our ways are not God's ways. We find an interesting story in the case of Brigid, a woman born in the early 400's in Ireland.
Brigid was born from a sexual encounter between an Irish king and one of his slaves. She was raised as a slave girl within the king's household and was required to perform hard work on the king's farm. From the beginning, Brigid took notice of the plight of the less fortunate. She would give the butter from the king's kitchen to working boys. She once gave the king's sword to a passing leper-an act about which the king was enraged. The king tried to marry her off, but to no avail. One day, Brigid fled the king's house and committed herself to belonging only to Christ.\\
\\
\noindent Brigid sought other women who also wanted to belong only to Christ. Seven of them organized a community of nuns that became known as the settlement of Kildare, a place where many thatch-roofed dwellings were built, and where artist studios, workshops, guest chambers, a library, and a church evolved. These and other settlements became little industries all to themselves, producing some of the greatest craftsmanship in all of Europe. Many of the poor had their lives bettered because of Brigid's ministry to them.\\
\\
\noindent Brigid became a traveling evangelist, helping the poor and preaching the gospel. When she died in 453, it is estimated 13,000 people had escaped from slavery and poverty to Christian service and industry. Her name became synonymous with the plight of the poor. She was a woman who turned a life of slavery and defeat into a life lived for a cause greater than herself. She became a nationally known figure among her people, and the Irish people still recognize her each February 1.\\
\\
\noindent God has called each of us to live for a cause greater than ourselves. If God asked you what you had done for the poor, what would you say? Jesus had a special place in His heart for the poor. Ask God how you might use your gifts and talents to improve the plight of the poor in your community.


