\section{Proverb 26 Comments}

\subsection{Numeric Nuggets}
Verse 24 has 13 words.  Verses 10, 23, and 24 have 13 unique words. he word ``in'' is used 13 times n the chapter. The word ``and'' is found 11 times plus 2 times in italics.

\subsection{Proverb 26:4-5}
Do Proverbs 26:4 and 26:5 contradict? How can both verses be true? Proverbs has much to say about fools. They despise wisdom (Proverbs 1:7, 22, 10:21, 23:9); they are right in their own eyes (Proverbs 12:15); they are deceitful (Proverbs 14:8) and scornful (Proverbs 10:23, 14:9). The wise are also given instruction on how to deal with fools in Proverbs. Instructing a fool is pointless because his speech is full of foolishness (Proverbs 15:2, 14) and he does not want wisdom and understanding (Proverbs 18:2).\\
\\
The futility of trying to impart wisdom to a fool is the basis of Proverbs 26:4-5, which tell us how to answer a fool. These seemingly contradictory verses are actually a common form of parallelism found in the Old Testament, where one idea builds upon another. Verse 4 warns against arguing with a fool on his own terms, lest we stoop to his level and become as foolish as he is. Because he despises wisdom and correction, the fool will not listen to wise reason and will try to draw us into his type of argument, whether it is by using deceit, scoffing at our wisdom, or becoming angry and abusive. If we allow him to draw us into this type of discourse, we are answering him ``according to his folly'' in the sense of becoming like him.\\
\\
The phrase ``according to his folly'' in verse 5, on the other hand, tells us that there are times when a fool has to be addressed so that his foolishness will not go unchallenged. In this sense answering him according to his folly means to expose the foolishness of his words, rebuking him on the basis of his folly so he will see the idiocy of his words and reasoning. Our ``answer'' in this case is to be one of reproof, showing him the truth so he might see the foolishness of his words in the light of reason. Even though he will most likely despise and reject the wisdom offered to him, we are to make the attempt, both for the sake of the truth which is always to be declared, and for the sake of those listening, that they may see the difference between wisdom and folly and be instructed.\\
\\
Whether we use the principle of verse 4 and deal with a fool by ignoring him, or obey verse 5 and reprove a fool depends on the situation. In matters of insignificance, it's probably better to disregard him. In more important areas, such as when a fool denies the existence of God (Psalm 14:1), verse 5 tells us to respond to his foolishness with words of rebuke and instruction. To let a fool speak his nonsense without reproof encourages him to remain wise in his own eyes and possibly gives credibility to his folly in the eyes of others.\\
\\
In short, in negligible issues we should just ignore fools, but in issues that matter, they must be dealt with so that credence will not be given to what they say. %https://www.gotquestions.org/Proverbs-26-4-5.html, 26 June 2020

\subsection{Proverb 26:16}
There are those who will not ever listen to reason. With seven being the number of completion, there are those who will not listen to the complete body of reason on a matter. According to verses 4 and 5, when this state is reached, don't bother any more. % Added 26 October 

\subsection{Proverb 26:17}
This is where the phrase ``It's not your problem'' may get it's inspiration.
\subsubsection{Meddle, meddling,  meddleth, intermeddle and intermeddleth}

Forms of the word ``meddle'' are used 12 times in the King James, only in the Old Testament.  For the most part, meddling with things is spoken against. The general rule is to let God take care of the problem.

\begin{compactenum}[1.][10]
	\item  \textbf{Deuteronomy 2:5} \fcolorbox{bone}{bone}{Meddle} not with them; for I will not give you of their land, no, not so much as a foot breadth; because I have given mount Seir unto Esau for a possession.
	\item \textbf{Deuteronomy 2:19} And when thou comest nigh over against the children of Ammon, distress them not, nor \fcolorbox{bone}{bone}{meddle} with them: for I will not give thee of the land of the children of Ammon any possession; because I have given it unto the children of Lot for a possession.
	\item \textbf{2 Kings 14:10} Thou hast indeed smitten Edom, and thine heart hath lifted thee up: glory of this, and tarry at home: for why shouldest thou \fcolorbox{bone}{bone}{meddle} to thy hurt, that thou shouldest fall, even thou, and Judah with thee?
	\item \textbf{2 Chronicles 25:19} Thou sayest, Lo, thou hast smitten the Edomites; and thine heart lifteth thee up to boast: abide now at home; why shouldest thou \fcolorbox{bone}{bone}{meddle} to thine hurt, that thou shouldest fall, even thou, and Judah with thee?
	\item \textbf{2 Chronicles 35:21} But he sent ambassadors to him, saying, What have I to do with thee, thou king of Judah? I come not against thee this day, but against the house wherewith I have war: for God commanded me to make haste: forbear thee from \fcolorbox{bone}{bone}{meddling} with God, who is with me, that he destroy thee not.
	\item \textbf{Proverb 14:10} The heart knoweth his own bitterness; and a stranger doth not \fcolorbox{bone}{bone}{intermeddle} with his joy.
	\item \textbf{Proverb 17:14} The beginning of strife is as when one letteth out water: therefore leave off contention, before it be \fcolorbox{bone}{bone}{meddled} with.
	\item \textbf{Proverb 18:1} Through desire a man, having separated himself, seeketh and \fcolorbox{bone}{bone}{intermeddleth} with all wisdom.
	\item \textbf{Proverb 20:3} It is an honour for a man to cease from strife: but every fool will be \fcolorbox{bone}{bone}{meddling}.
	\item \textbf{Proverb 20:19} He that goeth about as a talebearer revealeth secrets: therefore \fcolorbox{bone}{bone}{meddle} not with him that flattereth with his lips.
	\item \textbf{Proverb 24:21} My son, fear thou the LORD and the king: and \fcolorbox{bone}{bone}{meddle} not with them that are given to change:
	\item \textbf{Proverb 26:17} He that passeth by, and \fcolorbox{bone}{bone}{meddleth} with strife belonging not to him, is like one that taketh a dog by the ears.
\end{compactenum}
 % Added 26 October 2020
