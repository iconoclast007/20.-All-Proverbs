\section{Proverb 28 Outlines}

\subsection{My Outlines}


\subsubsection{Recognizing the Wicked}
\textbf{Introduction: }The practices and destiny of the wicked are fairly easy to recognize and fairly clear from scripture.%\footnote{28 October 2014, Keith Anthony}
\index[speaker]{Keith Anthony!Proverb 28 (Recognizing the Wicked)}
\index[series]{Proverbs (Keith Anthony)!Pro 28 (Recognizing the Wicked)}
\index[date]{2014/10/28!Proverb 28 (Recognizing the Wicked) (Keith Anthony)}

\begin{compactenum}[I.]
    \item \textbf{The Wicked Flee} They flee without being pursued, always afraid that their misdeeds are going to catch them. \index[scripture]{Proverbs!Pro 28:01}(Pro 28:1) 
    \item \textbf{The Wicked Forget} They forsake the law and are praised by other wicked for doing so. \index[scripture]{Proverbs!Pro 28:04}(Pro 28:4) 
    \item \textbf{The Wicked Forsakes his Father} \index[scripture]{Proverbs!Pro 28:07}(Pro 28:7)
    \item \textbf{The Wicked Fleece the Poor} \index[scripture]{Proverbs!Pro 28:08}(Pro 28:8) The wicked make their substance by usury and unjust gain, by cheating and dealing falsely.
    \item \textbf{The Wicked Fall into Their Own Pit} \index[scripture]{Proverbs!Pro 28:10}(Pro 28:10)
    \item \textbf{The Wicked are Foolish} \index[scripture]{Proverbs!Pro 28:16}(Pro 28:16)
    \item \textbf{The Wicked Follows his Own Heart} \index[scripture]{Proverbs!Pro 28:26}(Pro 28:26)
\end{compactenum}

\subsubsection{The Ways of  the Wicked}
Expressed in different terms, these are some of the ways of the wicked. Just in case we had any doubt of people who exhibit these behaviors, God calls them wicked. A wicked man:
\index[speaker]{Keith Anthony!Proverb 28 (The Ways of  the Wicked)}
\index[series]{Proverbs (Keith Anthony)!Pro 28 (The Ways of  the Wicked)}
\index[date]{2014/10/28!Proverb 28 (The Ways of  the Wicked) (Keith Anthony)}

\begin{compactenum}[I.]
    \item \textbf{Perverts the Way of Right} - The wicked rich get to their goal, riches, by perverting the ways of the Lord. But their riches do not make them better. \index[scripture]{Proverbs!Pro 28:06}(Pro 28:6)
    \item \textbf{Oppresses the Poor} - God has a special place in his heart and his eyes are always watching over the poor and helpless.  So he notices when they are mistreated.\index[scripture]{Proverbs!Pro 28:08}(Pro 28:8) 
    \item \textbf{Promotes only His Own Gain} - Is absolutely blind to the needs of others. \index[scripture]{Proverbs!Pro 28:11}(Pro 28:11) 
    \item \textbf{Pretends to be Righteous}  \index[scripture]{Proverbs!Pro 28:13}(Pro 28:13) 
    \item \textbf{Pursues Wealth blindly} - It is blind ambition. \index[scripture]{Proverbs!Pro 28:22}(Pro 28:22)
    \item \textbf{Pilfers from his Parents} - The attitude is one that the parents owe the money to him anyway.  The parents have been mistreating him by not financing every fleeting desire. \index[scripture]{Proverbs!Pro 28:24}(Pro 28:24)
    \item \textbf{Perish in Oblivion} - Precious in the eyes of the Lord is the death of his saints, but what of the wicked? \index[scripture]{Proverbs!Pro 28:28}(Pro 28:28)
\end{compactenum}

\subsubsection{Four Kinds of Men}
\textbf{Introduction: }There are four kinds of people described in Proverb 28. The chapter provide a rich description of each. Which one do you think you are?%\footnote{28 October 2014, Keith Anthony}
\index[speaker]{Keith Anthony!Proverb 28 (Four Kinds of Men)}
\index[series]{Proverbs (Keith Anthony)!Pro 28 (Four Kinds of Men)}
\index[date]{2018/01/28!Proverb 28 (Four Kinds of Men) (Keith Anthony)}
%\begin{compactenum}[1.]
%	\item Sunday, 28 Jan 2018, Pristine Rest Home, Xenia OH
%\end{compactenum}
\begin{compactenum}[I.]
    \item The \textbf{Rebellious} Man \index[scripture]{Proverbs!Pro 28:01}(Proverb 28:01) 
    \begin{compactenum}[A.]
    	\item the Rebellious man flees when no one pursues (vs 1)
    	\item the Rebellious man oppresses the poor (vs 3)
    	\item the Rebellious man does not understand judgment (v 5)
    	\item the Rebellious man is perverse in his way (v 6)
    	\item the Rebellious man has riotous people as friends (v 7)
    	\item the Rebellious man gets rich by usury and unjust gain (v 8)
    	\item the Rebellious man turns from the truth (v 9)
    	\item the Rebellious likes to corrupt good  people (v 10)
    	\item the Rebellious man covers his sins (v 13)
    	\item the Rebellious man hardens his heart (v 14)
    	\item the Rebellious man is a wicked ruler (v 15)
    	\item the Rebellious man lacks understanding (v 16)
    	\item the Rebellious man does violence (v 17)
    	\item the Rebellious man follows vain persons (v 18)
    	\item the Rebellious man looks for get rich quick schemes (v 20, 22)
    	\item the Rebellious man is a respecter of persons (v 21)
    	\item the Rebellious man robs his father and mother (v 24)
    	\item the Rebellious man trusts i his own heart (v 26)
    \end{compactenum}
Scripture is clear about rebellion and the rebellious
    \begin{compactenum}[1.]
    	\item 1 Samuel 15:23 says For rebellion is as the sin of witchcraft, and stubbornness is as iniquity and idolatry. Rebellion is what caused Saul to be removed as king.
    	\item In Job 34:37, Elihu says For he addeth rebellion unto his sin, he clappeth his hands among us, and multiplieth his words against God.
    	\item Proverbs 17:11 says An evil man seeketh only rebellion: therefore a cruel messenger shall be sent against him
	\end{compactenum}
	   \item The \textbf{Righteous} Man \index[scripture]{Proverbs!Pro 28:01}(Pro 28:01) 
    \begin{compactenum}[A.]
    	\item the Righteous man has boldness (vs 1)
    	\item the Righteous stands against wrong (v 4)
    	\item the Righteous man has understanding (v 5)
    	\item the Righteous man has true riches (v 6)
    	\item the Righteous man is wise (v 7)
    	\item the Righteous man shall have good possessions (v 10)
    	\item the Righteous man receives mercy (v 13)
    	\item the Righteous man has long life (v 16)
    	\item the Righteous man shall be saved (v 18)
    	\item the Righteous man knows the value of work (v 19)
    	\item the Righteous man gets blessings (v 20)
    	\item the Righteous man talks straight (v 23)
    	\item the Righteous man walks circumspectly (v 26)
    	\item the Righteous man is generous (v 27)
	\end{compactenum}
    \item The \textbf{Ruined} Man \index[scripture]{Proverbs!Pro 28:10}(Pro 28:10) 
    \begin{compactenum}[A.]
    	\item the Ruined man has turned from the truth (vs 4, 9)
    	\item the Ruined man has listened to a Rebellious One (vs 10)
    	\item the Ruined man got there by covering his Sins (vs 13)
    	\item the Ruined man has hardened his heart (vs 14)
    	\item the Ruined man has become perverse (vs 18)
    	\item the Ruined man has followed shortcuts (vs 22)
    	\item the Ruined man has trusted his own heart (vs 26)\newline
	\end{compactenum}
Isaiah 3:8 gives us a description of how to be ruined, speaking of Israel: For Jerusalem is ruined, and Judah is fallen: because their tongue and their doings are against the LORD, to provoke the eyes of his glory.
    \item The \textbf{Rich} Man \index[scripture]{Proverbs!Pro 28:11}(Pro 28:11) 
    \begin{compactenum}[A.]
    	\item the Rich man can be good or perverse (vs 6)
    	\item the Rich man sometimes gets there by being unjust (v 8)
    	\item the Rich man can be wise in his own conceits (v 11)
    	\item the Rich man can have an evil eye (v 22)
	\end{compactenum}
Riches are empty and unsatisfying, and can be used for evil or for good. They are not an end to themselves. There are better things:
    \begin{compactenum}[A.]
    	\item Pro 15:16 says Better is little with the fear of the LORD than great treasure and trouble therewith.
    	\item Pro 15:17 says Better is a dinner of herbs where love is, than a stalled ox and hatred therewith.
    	\item Pro 16:8 says Better is a little with righteousness than great revenues without right.
    	\item Pro 16:16 says How much better is it to get wisdom than gold! and to get understanding rather to be chosen than silver! 
    	\item Pro 19:1 Better is the poor that walketh in his integrity, than he that is perverse in his lips, and is a fool.
    	\item Eccl 4:13 says Better is a poor and a wise child than an old and foolish king, who will no more be admonished.
	\end{compactenum}

\end{compactenum}
\textbf{Summary:} But there is a fifth kind of person, the Redeemed
