\section{Proverb 9 Comments}

\subsection{Numeric Nuggets}
\textbf{13:} The 13-letter word ``understanding'' is used in Proverb 9. Verses 5 and 12 have 13 unique words. The word ``and'' is used 13 times, one instance in italics.

\subsection{Proverb 9:1}
What about these seven pillars?
\begin{compactenum}
    \item William Grady maps the seven pillars of wisdom to 2 Peter 2:4-7: (1) faith, (2) virtue, (3) knowledge, (4) temperance, (5) godliness, (6) brotherly kindness, and (7) charity. These pillars would map to seven physical pillars around the temple: 3 of the south side, temperance on the west, and the remaining three heading back toward the Temple opening on the north \cite{grady2010given}. This is the interpretation which I think fits best.
    \item Henry Morris, of the Institute for Creation Research (ICR), has the seven pillars listed in James 3:17:  (1) genuine purity, (2) peaceableness, (3) gentleness, (4) reasonableness, (5) helpfulness, (6) humility, and (7) sincerity. But these words are not found in the AV1611.  Instead are adjectives describing wisdom.
    \item Some equate these 7 pillars withe seven divisions of Proverbs: identified in (1) Proverbs 1:1 "The proverbs of Solomon", (2) Proverbs 10:1 "The proverbs of Solomon", (3) Proverbs 22:17 "hear the words of the wise", (4) Proverbs 24:23 "These things also belong to the wise", (5) Proverbs 25:1 "These are also the proverbs of Solomon", (6) Proverbs 30:1 "The words of A'gur... even the prophecy...", and (7) Proverbs 31:1.
    \item But, noting the septatic (7) structure found throughout scripture, some equate the pillars to the seven main sections of the Bible: (1) Torah/Law,(2) OT History, (3) Wisdom Books, (4) Major Prophets, (5) Minor Prophets, (6) NT History, and (7) NT Epistles.
    \item William Brown, in The Seven Pillars of Creation: The Bible, Science, and the Ecology of Wonder, cites Rabbinical scholars who equated these pillars with the seven days of creation. \cite{brown2010SevenPillars}
\end{compactenum}

\subsection{Proverb 9:7, 8}
The verses say to not waste your time with a scorner.

\subsection{Proverb 9:13}
This is the only use of the word ``clamorous'' in the KJV. Webster's has three definitions of the word. First. Clamorous is one who ``engages in or is marked by loud and insistent cries especially of protest."  Clamorous is ``full of or characterized by the presence of noise.'' And clamorous is ``marked by a high volume of sound.''  A list of synonyms includes: blatant, caterwauling, clamant, obstreperous, squawking, vociferant, vociferating, vociferous, yawping (or yauping), yowling, clangorous, clattering, clattery, noisy, rackety, resounding, uproarious, blaring, blasting, booming, deafening, earsplitting, loud, piercing, plangent, resounding, ringing, roaring, slam-bang, sonorous, stentorian, thundering, and thunderous. 

\subsection{Proverb 9:15}
The word ``wanteth'' here. is understood as ``lacks.'' The foolish and clamorous woman is not speaking to the one with understanding, but the one who does not have any.

