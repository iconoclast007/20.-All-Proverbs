\section{Proverb 18 Comments}

\subsection{Numerical Nuggets}
\textbf{13:} The word ``wellspring'' found only twice in scripture is the 13$^{th}$ word in verse 4. Two thirteen-letter words are in the chapter: ``intermeddleth'' and ``understanding.'' Verse 12, speaking of haughtiness, has 13 words. Verses 1, 12, and 16 have 13 words. Verses 7 and 12 have 13 unique words. The word ``his'' is used 13 times in the chapter. The 13-letter words ``intermeddleth'' and ``understanding'' are used in the chapter.

\subsection{Proverb 18:1-2}
The issue addressed in verses 1 and 2 is the human heart, and as such it should be expected that man will confront and change it, as shown in Table~\ref{table:CorruptionProv18:1}. The verse goes to motive. This is why most (or maybe all) research, ostensibly done for the betterment of mankind, eventually is used to oppress mankind. It comes down to power, position, prestige, prominence, and power. It comes down to why one seeks those things. Verse 2 puts the thought in context: it is the fool who trusts his heart and seeks the heart's desire. Why is this man a fool? It is because a sincere, God-seeking man will quickly elan from scripture the sordid details and depravity of the human heart.\footnote{\textbf{Isaiah 5:21} - Woe unto them that are wise in their own eyes, and prudent in their own sight!}\footnote{\textbf{Luke 10:21} - In that hour Jesus rejoiced in spirit, and said, I thank thee, O Father, Lord of heaven and earth, that thou hast hid these things from the wise and prudent, and hast revealed them unto babes: even so, Father; for so it seemed good in thy sight.}\\
\\
\noindent One commentator explains the process \cite{ruckman1972proverbs}:
\begin{compactenum}[1.]
    \item DESIRE is at the root of all scientific research and investigation (verse 1),
    \item This desire comes form the human HEART and not the human MIND (verse 2),
    \item This desire is essentially subjective, egotistical, and self-seeking (verse 2),
    \item This desire leads a man to pursue the life of isolated superiority, perhaps even the ivory towers of academic, or the esteemed halls of theological institutions, where he deep-down considers himself to be above the masses, or above the common folk. This may even start out as, or continue to masquerade as, a noble cause.
    \item The man does not seek wisdom. He ``seeketh and intermeddleth with all wisdom.'' 
    \item The man's ``love of wisdom'' is neither pure, not objective, not philanthropic.
    \item The wisdom ``sought'' is complete and is a mean's to an end.
\end{compactenum}

\newpage
\begin{center}

\begin{table}[ht]
\centering
\begin{tabular}{|p{.5in}|p{3.5in}|}
    \hline
    \textcolor{blue}{AV} & \textcolor{blue}{Through desire a man, having separated himself, seeketh \emph{and} intermeddleth with all wisdom.}\\ \hline
    \hline
    CEB &  Unfriendly people look out for themselves; they bicker with sensible people.\\ \hline
%
ESV & Whoever isolates himself seeks his own desire;  he breaks out against all sound judgment. \\ \hline
%
NASV &  He who separates himself seeks his own desire, He quarrels against all sound wisdom.\\ \hline
%
MEV & He who separates himself seeks his own desire; he seeks and quarrels against all wisdom.\\ \hline
%
NIV &  An unfriendly person pursues selfish ends and against all sound judgment starts quarrels. \\ \hline
%
NKJV &  A man who isolates himself seeks his own desire; He rages against all wise judgment.\\ \hline
%
RSV &  He who is estranged seeks pretexts  to break out against all sound judgment.\\ \hline \hline

\multicolumn{2}{|p{4.2in}|}{{\textcolor{jungle}{Modern translations, such as the ASV and others, strike out the first part of the verse, concealing the intent of mankind in general.  It goes to the heart's intent in seeking wisdom. How wonderful is the obfuscated RSV text: ``He who is estranged seeks pretexts.'' What does THAT mean?}}} \\ \hline

\end{tabular}
\caption[Corruption Alert: Proverb 18:1]{Corruption Alert: Proverb 18:1} \label{table:Corruption Proverb 18:1}
\end{table}

\end{center}


\newpage

