\section{Proverb 18 Outlines}

\subsection{My Outlines}

\subsubsection{A Man Focused on God's Wisdom}
\index[speaker]{Keith Anthony!Proverb 18 (A Man Focused on God's Wisdom)}
\index[series]{Proverbs (Keith Anthony)!Pro 18 (A Man Focused on God's Wisdom)}
\index[date]{2014/11/18!Proverb 18 (A Man Focused on God's Wisdom) (Keith Anthony)}
Proverbs 18:1 gives some hints about a man who is ostensibly focused on getting God's wisdom. Can we relate to this kind of person? Do we have any biblical examples? Could it be Elijah, or John the Baptist?
\begin{compactenum}[I.]
    \item Will be an \textbf{Inspired Man} - 
    \item Will be \textbf{Separate from the Ignorant Masses} - 
    \item For him most things in life will have  \textbf{Insignificant Meaning} - 
    \item Lives in an \textbf{Isolated Manner} 
    \item Lives by an \textbf{Individual Mandate} -  seldom supported by friends, and 
    \item Has an \textbf{intense and Insular Mentality} - 
\end{compactenum}
So, this is a guy who is mostly unwanted and unwelcome. Vast majority cannot related to him. How much are you like this man? How much do you want to be? How much do you think you should be?

\subsubsection{A Friend Like Jesus}
\index[speaker]{Keith Anthony!Proverb 18 (A Friend Like Jesus)}
\index[series]{Proverbs (Keith Anthony)!Pro 18 (A Friend Like Jesus)}
\index[date]{2014/11/18!Proverb 18 (A Friend Like Jesus (Keith Anthony)}
\begin{compactenum}[I.][8]
    \item A \textbf{Close} Friend
    \item A \textbf{Constant} Friend
    \item A \textbf{Confiding} Friend
    \item A \textbf{Concerned} Friend
    \item A \textbf{Continuing} Friend
    \item A \textbf{Capable} Friend
    \item A \textbf{Caring} Friend
\end{compactenum}


\subsubsection{The Pursuit of Wisdom}
\index[speaker]{Keith Anthony!Proverb 18 (The Pursuit of Wisdom)}
\index[series]{Proverbs (Keith Anthony)!Pro 18 (The Pursuit of Wisdom)}
\index[date]{2017/03/18!Proverb 18:01 (The Pursuit of Wisdom) (Keith Anthony)}
At issue in the pursuit of wisdom, is the underlying heart motive. I contest that ``Wisdom'' is not a thing, but a person, an individual, who is reading our motives as we go. I also contest that going to wisdom with the wrong motive could lead to disastrous results. Motives range from 100 per cent evil, to (seldom if ever) 100 per cent right. Some of these motives:
\begin{compactenum}[I.]
    \item Then \textbf{Perverted} Motive -- to learn how manipulate events and circumstances. \index[scripture]{Proverbs!Pro 18:01}(Proverb 18:1)
    \item Then \textbf{Prideful} Motive -- To become exalted as wise. 
    \item Then \textbf{Public} Motive -- to be known and famous. 
    \item Then \textbf{Personal} Motive -- to achieve understanding. 
    \item Then \textbf{Practical} Motive -- to learn how live without mistakes or errors. 
    \item Then \textbf{Purposeful} Motive -- to learn navigate a specific problem. 
    \item Then \textbf{Perfect} Motive -- to learn how to live righteously before God. 
\end{compactenum}

\subsubsection{What Your Speech Reveals}
My other main text will be James 3:3-10. which I will quote:  Behold, we put bits in the horses’ mouths, that they may obey us; and we turn about their whole body. [4] Behold also the ships, which though they be so great, and are driven of fierce winds, yet are they turned about with a very small helm, whithersoever the governor listeth. [5] Even so the tongue is a little member, and boasteth great things. Behold, how great a matter a little fire kindleth! [6] And the tongue is a fire, a world of iniquity: so is the tongue among our members, that it defileth the whole body, and setteth on fire the course of nature; and it is set on fire of hell. [7] For every kind of beasts, and of birds, and of serpents, and of things in the sea, is tamed, and hath been tamed of mankind: [8] But the tongue can no man tame; it is an unruly evil, full of deadly poison. [9] Therewith bless we God, even the Father; and therewith curse we men, which are made after the similitude of God. [10] Out of the same mouth proceedeth blessing and cursing. My brethren, these things ought not so to be. [11] Doth a fountain send forth at the same place sweet water and bitter? 12 Can the fig tree, my brethren, bear olive berries? either a vine, figs? so can no fountain both yield salt water and fresh.\\
\\
Consider the verses Proverbs 18:4, Proverbs 18:6, Proverbs 18:7, Proverbs 18:8, and Proverbs 18:13.  These verses say a lot about what people say. What does your speech reveal? Listen to someone long enough and their tongue will betray their innermost secrets ... The things that they love... The things that they focus on... It will reveal if they swim with the deep thinkers or wade in the shallow end of the pool with the other immature and self-focused babies ... Your concept of God, Christianity, and spiritual truth will eventually be reflected in your speech, in casual conversation ... Matthew 26:73, speaking of Peter's denial of Christ, says: And after a while came unto him they that stood by, and said to Peter, Surely thou also art one of them; for thy speech bewrayeth thee.\\
\\
Some things your tongue will tell about you:
\index[speaker]{Keith Anthony!Proverb 18:04 (What Your Speech Reveals)}\\
\index[series]{Proverbs (Keith Anthony)!Pro 18 (What Your Speech Reveals)}
\index[date]{2014/10/18!Proverb 18 (What Your Speech Reveals) (Keith Anthony)}

\index[LOCATION]{Greene County Adult Detention Center!2022/01/18!Tuesday Night}

\begin{compactenum}[I.][10]
    \item \textbf{Your Education} Do you use bad grammar? Intentionally?  Do you use Ebonics?  Is every word, or every other word, out of your mouth a cuss word?  When I say, though, what I really mean is wisdom. 
    \item \textbf{Your Excellence} Or lack of excellence ... If you are a child of God, you are a child of THE King and should talk like it. Back it Daniel 6:3 scripture says, ``Then this Daniel was preferred above the presidents and princes, because an excellent spirit was in him; and the king thought to set him over the whole realm.'' Proverbs 8:6 says: ``Hear; for I will speak of excellent things; and the opening of my lips shall be right things.''
    \item \textbf{Your Excitements} -- what gets you going
    \item \textbf{Your Enmity} -- So, you just can't help with this one ... At some point you're going to speak up on the things that irritate you 
    \item \textbf{Your Enemies} -- who you just might hurt if you had the chance.. But scripture says to let the Lord have vengeance. Nahum 1:2 says: God is jealous, and the LORD revengeth; the LORD revengeth, and is furious; the LORD will take vengeance on his adversaries, and he reserveth wrath for his enemies.
    \item \textbf{Your Enticements} What about that new Corvette?  How about that new Tesla.  I'd sure have fun shooting an AR-15 ... Whata about that new Movie ... and the lead actress?
    \item \textbf{Your Entanglements} -- what preoccupies you is what has you in chains
    \item \textbf{Your Emptiness} A while ago I saw an old friend, who attends a church I used to go to, and who I had not seen in about 5 years. I was expecting a warm ``glad to see you'', but instead got some comment about still being overweight. I guess that gives a measure of that friendship.  Getting more personal, ow many times have we asked ``how are you?'' when we really were not interested in the answer? We spend a great ... Do you or I speak of things that have substance? The problem is that the unsaved really have nothing of substance.
   \item \textbf{Your End} considering the rest of Proverbs, it if usually clear where your road is headed ... Christians often, sometimes usually, talk about heaven.  The lost talk about everything except heaven.  
\end{compactenum}
Matthew 12:36-37 tells us: ``But I say unto you, That every idle word that men shall speak, they shall give account thereof in the day of judgment. [37] For by thy words thou shalt be justified, and by thy words thou shalt be condemned.''\\
\\
So, (1) Consider, (2) Correct, (3) Covert, (4) Control, (5)

\subsubsection{The Wise Man}
\index[speaker]{Keith Anthony!Proverb 18 (The Wise Man)}
\index[series]{Proverbs (Keith Anthony)!Pro 18 (The Wise Man)}
\index[date]{2014/11/18!Proverb 18 (The Wise Man) (Keith Anthony)}
Proverbs 18:1 gives some hints about a wise man:
\begin{compactenum}[I.]
    \item \textbf{Is Distinctive}
    \item \textbf{Is Disciplined}
    \item \textbf{Ignores Distractions}
    \item \textbf{Finds Delights in Wisdom}
    \item \textbf{Has Departed from the Common and Mundane}
    \item \textbf{Has a Distaste for Foolishness}
    \item \textbf{Is often regarded as Disturbed}
\end{compactenum}

